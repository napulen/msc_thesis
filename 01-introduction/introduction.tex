\normallinespacing

\chapter{Introduction}
One of the most important books in western music history, and considered to be the first harmony book by many music theorists is Jean Philippe Rameau’s “Treatise on harmonie”, published in 1722 CITE{beach}. Previous to this vertical vision of music composition, there was counterpoint, the horizontal vision of music. Not only is this vertical or horizontal paradigm important for writing, but also for understanding and explaining how music works in general. Specially in music composition, harmony could be considered one of the pillars of western classical music. Even in the sophisticated counterpoint writing of the early eighteenth century, we could say harmonic principles accompany the evolution of melodic lines, and lead the music to triads and cadences. In fact, in the words of Jean-Philippe Rameau, harmony comprehends melody CITE{rameau}.

Once established, the harmonic paradigm remained active until the end of the 19th century. Throughout the years, there have been several ways to approach and understand the harmony theory. The original idea by Rameau was based on the movement of the bass note, the so called "Fundamental Bass", which constructed rules and principles of how the movement of a bass note determined also the movement of the harmony. Following to that, Joseph Vogler and Gottfried Weber developed the concept of "Multiple Meaning", which considered the meaning of chords was dependent on the musical context, introducing also the roman numeral labels in the musical analysis. Later on, Hugo Riemann presented a third approach which restricted the study of harmony to only three functions (Tonic, SubDominant and Dominant), where every degree belongs to one of these functions. This approach is commonly known as "Functional Harmony". This work deals with the automatic analysis of harmony based on this third point of view.

\section{Motivation}
Harmony is not explicitly written in music scores. However, even with its implicit nature, harmony can be annotated in a music score by using music notation conventions \cite{harte}, we usually refer to this as harmonic analysis. We want that harmonic analysis to be annotated automatically to symbolic music representations.

With this annotations, we make more evident to the listener how the harmony is influencing the evolution and movement of the music. It also helps in other analysis tasks such as finding musical structure. Similarly, we could help the music student providing some aid or initial guess of the harmonic structure of a piece of music. Facilitating the harmonic analysis and the harmonic reduction of a piece of music.

\section{Objectives}
This work pretends to reproduce and apply the current approach for automatic functional harmonic analysis used over the KernScores corpora, and apply it to a specific set of string quartets from Joseph Haydn. Once these automatic analyses are performed, they will be compared to manual annotations over the same set of string quartets.

\section{Structure of the Report}


\newpage


