\chapter{Introduction}
What is harmonic analysis about.
In the traditional sense, harmonic analysis relates to a set of music theory practices that pretend to describe the relationship of vertical sonorities, generalize rules of how each of the horizontal elements should move to facilitat and embellish the vertical sonorities, it could be said to be a very theoretical field of western music. Harmonic analysis could also be thought as a cognitive process that happens to listeners familiarized to western music.

\section{Motivation}
Why is it important to make it automatic.
The main reason why it would be benefical for musicians and musicologists to automate a harmonic analysis is because it takes a considerable amount of time and knowledge to perform these analyses. In practice, students at conservatoires often learn the guidelines of harmonic analysis during a specialized course of Harmony. Such a course could extend to several years, making it a difficult discipline to learn fast for a beginner. Even in the case of experts, it might take relatively long time to analyze a piece of music in terms of harmony, which makes the task of automating it valuable even for the expert theorist.

\section{Objectives}
This work pretends to reproduce and apply the current approach for automatic functional harmonic analysis used over the KernScores corpora, and apply it to a specific set of string quartets from Joseph Haydn. Once these automatic analyses are performed, they will be compared to manual annotations over the same set of string quartets.

\section{Structure of the Report}
Literature. Methods. Results. Discussion.

\newpage
