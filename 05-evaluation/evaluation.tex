\chapter{Evaluation}
\label{chap:evaluation}
From all the implications of this work, probably the evaluation is the one that adds more value to the task of automatic harmonic analysis.

% If one reviews the works of automatic harmonic analysis, since the pioneer attempt of Winograd \cite{winograd1968linguistics} until David Temperley's \cite{temperley1997algorithm}, and even the ones from Christopher Raphael \cite{raphael2003harmonic} and Pl\'acido Illescas \cite{illescas2007harmonic}, it is visible that almost none of the efforts proposes any quantitative measure of how accurate the resulting analysis is. This is easy to understand, because evaluating a harmonic analysis is very difficult, even in the case of human analysts, one could have very different answers for the same piece of music, and both could be correct. Therefore, it is understandable that researchers have not put themselves in the necessity of validating how good or bad is the roman numeral label annotated automatically.

During this chapter, I will introduce the evaluation process that I propose for validating how similar a manually annotated harmonic analysis stands compared to another generated automatically using the methdods explained during \autoref{chap:methodology}.

\section{After the automatic analysis}
So far, I reviewed the implementation efforts that materialize a system of automatic harmonic analysis which takes a musical score encoded in humdrum and outputs a similar musical score with roman numeral labels in the **harm syntax appended to it.

I also described a dataset of manual annotations in the **harm syntax.

Now, I will describe the process of putting together both, the manually annotated score and the automatic analysis into a single file. I call this file, the evaluation file.

\section{Evaluation file}
	The evaluation file is a valid humdrum file with 4 spines.

	The first pair of spines represents the roman numeral and harmonic root of the manual analysis, while the second represents the corresponding elements of the automatic analysis.

	\begin{table}[tbp]
	\centering
	\begin{tabular}{|ll|ll|}
	\hline
	\multicolumn{2}{|l|}{Manual analysis} & \multicolumn{2}{l|}{Automatic analysis} \\ \hline
	**harm & **root & **harm & **root \\
	. & . & . & . \\
	. & . & . & . \\
	. & . & . & . \\
	. & . & . & . \\
	. & . & . & . \\
	. & . & . & . \\
	1 & 1 & 1 & 1 \\
	i & g & . & . \\
	. & . & . & . \\
	iio & a & iv & f \\
	. & . & . & . \\
	ib & g & . & . \\
	. & . & . & . \\
	2 & 2 & 2 & 2 \\
	iv & c & ib & c \\
	. & . & . & . \\
	. & . & V7c & g \\
	. & . & . & . \\
	ib & g & . & . \\
	. & . & . & . \\ \hline
	\end{tabular}
	\caption{Sample evaluation file from Op.20 No.3 - II., mm.1-2}
	\label{table:evaluation_file}
	\end{table}

	\autoref{table:evaluation_file} summarizes this explanation with a fragment of an evaluation file.

  \subsection{Generating}
  Take the manual and automatic, extract the harmonic root using harm2kern, append and filter
  \subsection{Rhythmic normalization}
  Files are normalized to the shortest note in the score, specially needed for dealing with polyrhythms
	\subsection{Using Humdrum-extras}
  Using harm2kern to extract the harmonic root, for now, that is enough, might change in the future.
\section{Comparing}
	\subsection{Resolving root from **harm expression}
  harm2kern for now
	\subsection{Ignored annotations}
  Ignoring 'Chr' annotations. See the issues section.
\section{Extracting final results}
	\subsection{Total time units}
  The time unit represents the shortest note for that particular score. The scores are normalized to that note as the rhythmic "atom", irrelevant of its value.
	\subsection{Percentage}
  Every time unit is matched between manual and automatic file, the percentages are computed considering the total number of "time units" in the score, and the ones that match identically in harmonic root between both files.
	\subsection{Distribution of degrees}
  Additional statistics are provided of how the tonal degrees distribute per each score in both, manual and automatic files.
	\subsection{Resolution of degree in secondary functions}
  This is a special problem to consider when evaluating functional harmonic analysis. Every subfunction represents a diatonic degree of the current key, an uppercase diatonic degree represents the major mode tonality of that degree, a lowercase diatonic degree represents the minor mode tonality of that degree.

\newpage
