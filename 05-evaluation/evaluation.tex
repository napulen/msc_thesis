\chapter{Evaluation}
\label{chap:evaluation}
From all the implications of this work, probably the evaluation is the one that contributes some novelty to the task of automatic harmonic analysis.

% If one reviews the works of automatic harmonic analysis, since the pioneer attempt of Winograd \cite{winograd1968linguistics} until David Temperley's \cite{temperley1997algorithm}, and even the ones from Christopher Raphael \cite{raphael2003harmonic} and Pl\'acido Illescas \cite{illescas2007harmonic}, it is visible that almost none of the efforts proposes any quantitative measure of how accurate the resulting analysis is. This is easy to understand, because evaluating a harmonic analysis is very difficult, even in the case of human analysts, one could have very different answers for the same piece of music, and both could be correct. Therefore, it is understandable that researchers have not put themselves in the necessity of validating how good or bad is the roman numeral label annotated automatically.

During this chapter, I will introduce the evaluation process that I propose for validating how similar a manually annotated harmonic analysis stands compared to another generated automatically using the methods explained during \autoref{chap:methodology}.

\section{After the automatic analysis}
So far, I reviewed the implementation efforts that materialize a system of automatic harmonic analysis which takes a musical score encoded in humdrum and outputs a similar musical score with roman numeral labels in the **harm syntax appended to it.

One important thing to know is that, together with the manual annotations described during \autoref{chap:dataset}, all the automatic analyses for the musical scores used in this work have been also saved in the same repository where the manual annotations are contained.

This means, in summary, that the repository (\url{https://github.com/napulen/haydn_op20_harm}) where the dataset lives holds manual and automatic annotations for the musical scores.

Now, I will describe the process of putting together both, the manual and automatic harmonic analysis annotations into a single file. A single file that can then be used to compare the similarity of the analyses. I call this file, the evaluation file.

\section{Evaluation file}
	The evaluation file is a valid humdrum file with four spines.

	The first pair of spines represents the roman numeral and harmonic root of the manual analysis, while the second represents the corresponding elements of the automatic analysis.

	\begin{table}[tbp]
	\centering
	\begin{tabular}{|cc|cc|}
	\hline
	\multicolumn{2}{|c|}{Manual analysis} & \multicolumn{2}{c|}{Automatic analysis} \\ \hline
	**harm & **root & **harm & **root \\
	. & . & . & . \\
	. & . & . & . \\
	. & . & . & . \\
	. & . & . & . \\
	. & . & . & . \\
	. & . & . & . \\
	=1 & =1 & =1 & =1 \\
	i & g & . & . \\
	. & . & . & . \\
	iio & a & iv & f \\
	. & . & . & . \\
	ib & g & . & . \\
	. & . & . & . \\
	=2 & =2 & =2 & =2 \\
	iv & c & ib & c \\
	. & . & . & . \\
	. & . & V7c & g \\
	. & . & . & . \\
	ib & g & . & . \\
	. & . & . & . \\ \hline
	\end{tabular}
	\caption{Sample evaluation file.}
	\label{table:evaluation_file}
	\end{table}

	\autoref{table:evaluation_file} provides a first glance at a fragment of an evaluation file, extracted from the first two measures of the second movement of the String Quartet Op.20 No.3.

	The first characteristic I want to discuss about this evaluation file, is its normalization of time units.

  \subsection{Normalization of time units}
		%
		% \begin{table}[tbp]
		% \centering
		% \begin{tabular}{|c|c|c|c|c|}
		% \hline
		% Measure & \multicolumn{2}{c|}{Analysis 1} & \multicolumn{2}{c|}{Analysis 2} \\ \hline
		% \multirow{2}{*}{Anacrusis} & . & . & . & . \\ \cline{2-5}
		%  & . & . & . & . \\ \hline
		% \multirow{3}{*}{First measure} & i & g & . & . \\ \cline{2-5}
		%  & iio & a & iv & f \\ \cline{2-5}
		%  & ib & g & . & . \\ \hline
		% \multirow{3}{*}{Second measure} & iv & c & ib & c \\ \cline{2-5}
		%  & . & . & V7c & g \\ \cline{2-5}
		%  & ib & g & . & . \\ \hline
		% \end{tabular}
		% \caption{Original time units}
		% \label{table:no_normalization}
		% \end{table}
		%
		% If the evaluation file displayed before would keep the original time units of the labels, it would look like as shown in \autoref{table:no_normalization}.

		One problem that arises with the way **harm spines are appended to the scores, is that these spines do not store any explicit information about time duration. The labels are properly interpreted in the full score because they are aligned to the note values from the instruments, which do have time duration information.

		In order to avoid this and other problems related to duration, during the creation of the evaluation file, the time units of the score get normalized to the shortest note of the score.

	\subsection{Finding the shortest note}
		In order to find the shortest note, the \emph{census} program is being used, which is part of the \emph{Humdrum Toolkit} \cite{humdrum}.

		\autoref{table:census} shows the output of the \emph{census} program for the humdrum file of the second movement of Op.20 No.3. Among the output of the \emph{KERN DATA} section, the duration of the shortest note of the humdrum score is displayed. This note duration is parsed from the output of the program and used to make a new timebase for the humdrum file.

		\begin{table}[]
		\centering
		\begin{tabular}{ll}
		HUMDRUM DATA &  \\
		 &  \\
		Number of data tokens: & 1780 \\
		Number of null tokens: & 515 \\
		Number of multiple-stops: & 1 \\
		Number of data records: & 445 \\
		Number of comments: & 27 \\
		Number of interpretations: & 7 \\
		Number of records: & 479 \\
		 &  \\
		KERN DATA &  \\
		 &  \\
		Number of note-heads: & 785 \\
		Number of notes: & 741 \\
		Longest note: & 2 \\
		Shortest note: & 8 \\
		Highest note: & ddd \\
		Lowest note: & DD \\
		Number of rests: & 125 \\
		Maximum number of voices: & 5 \\
		Number of single barlines: & 88 \\
		Number of double barlines: & 1
		\end{tabular}
		\caption{Output of the census program}
		\label{table:census}
		\end{table}

		As a matter of explanation, the string value that appears as the shortest note is in the reciprocal notation for duration. In this case, a value of "8" means the sortest note is 1/8 of a whole note, or an eight note. With this said, if this example holds a time signature of 3/4, six eight notes are expected for every measure.

		% Please add the following required packages to your document preamble:
		% \usepackage{multirow}
		\begin{table}[tbp]
		\centering
		\begin{tabular}{|c|c|c|c|c|}
		\hline
		Measure & \multicolumn{2}{c|}{Analysis 1} & \multicolumn{2}{c|}{Analysis 2} \\ \hline
		\multirow{6}{*}{Anacrusis} & . & . & . & . \\ \cline{2-5}
		 & . & . & . & . \\ \cline{2-5}
		 & . & . & . & . \\ \cline{2-5}
		 & . & . & . & . \\ \cline{2-5}
		 & . & . & . & . \\ \cline{2-5}
		 & . & . & . & . \\ \hline
		\multirow{6}{*}{First} & i & g & . & . \\ \cline{2-5}
		 & . & . & . & . \\ \cline{2-5}
		 & iio & a & iv & f \\ \cline{2-5}
		 & . & . & . & . \\ \cline{2-5}
		 & ib & g & . & . \\ \cline{2-5}
		 & . & . & . & . \\ \hline
		\multirow{6}{*}{Second} & iv & c & ib & c \\ \cline{2-5}
		 & . & . & . & . \\ \cline{2-5}
		 & . & . & V7c & g \\ \cline{2-5}
		 & . & . & . & . \\ \cline{2-5}
		 & ib & g & . & . \\ \cline{2-5}
		 & . & . & . & . \\ \hline
		\end{tabular}
		\caption{Adding time normalization to the evaluation file}
		\label{table:normalization}
		\end{table}

		\autoref{table:normalization} shows precisely this, the normalized evaluation file to eight notes.

	\subsection{Using Humdrum-extras}
  Using harm2kern to extract the harmonic root, for now, that is enough, might change in the future.
\section{Comparing}
	\subsection{Resolving root from **harm expression}
  harm2kern for now
	\subsection{Ignored annotations}
  Ignoring 'Chr' annotations. See the issues section.
\section{Extracting final results}
	\subsection{Total time units}
  The time unit represents the shortest note for that particular score. The scores are normalized to that note as the rhythmic "atom", irrelevant of its value.
	\subsection{Percentage}
  Every time unit is matched between manual and automatic file, the percentages are computed considering the total number of "time units" in the score, and the ones that match identically in harmonic root between both files.
	\subsection{Distribution of degrees}
  Additional statistics are provided of how the tonal degrees distribute per each score in both, manual and automatic files.
	\subsection{Resolution of degree in secondary functions}
  This is a special problem to consider when evaluating functional harmonic analysis. Every subfunction represents a diatonic degree of the current key, an uppercase diatonic degree represents the major mode tonality of that degree, a lowercase diatonic degree represents the minor mode tonality of that degree.

\newpage
