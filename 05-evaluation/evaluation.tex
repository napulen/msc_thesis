\chapter{Evaluation}
\label{chap:evaluation}
From all the implications of this work, probably the evaluation is the one that contributes some novelty to the task of automatic harmonic analysis.

During this chapter, I will introduce the evaluation process that I propose for validating how similar a manually annotated harmonic analysis stands compared to another generated automatically using the methods explained during \autoref{chap:methodology}.

\section{After the automatic analysis}
So far, I reviewed the implementation efforts that materialize a system of automatic harmonic analysis which takes a musical score encoded in humdrum and outputs a similar musical score with roman numeral labels in the **harm syntax appended to it.

One important thing to know is that, together with the manual annotations described during \autoref{chap:dataset}, all the automatic analyses for the musical scores used in this work have been also saved in the same repository where the manual annotations are contained.

This means, in summary, that the repository (\url{https://github.com/napulen/haydn_op20_harm}) where the dataset lives holds manual and automatic annotations for the musical scores.

Now, I will describe the process of putting together both, the manual and automatic harmonic analysis annotations into a single file. A single file that can then be used to compare the similarity of the analyses. I call this file, the evaluation file.

\section{Evaluation file}
	The evaluation file is a valid humdrum file with four spines.

	The first pair of spines represents the roman numeral and harmonic root of the manual analysis, while the second represents the corresponding elements of the automatic analysis.

	\begin{table}[tbp]
	\centering
	\begin{tabular}{|cc|cc|}
	\hline
	\multicolumn{2}{|c|}{Manual analysis} & \multicolumn{2}{c|}{Automatic analysis} \\ \hline
	**harm & **root & **harm & **root \\
	. & . & . & . \\
	. & . & . & . \\
	. & . & . & . \\
	. & . & . & . \\
	. & . & . & . \\
	. & . & . & . \\
	=1 & =1 & =1 & =1 \\
	i & g & . & . \\
	. & . & . & . \\
	iio & a & iv & f \\
	. & . & . & . \\
	ib & g & . & . \\
	. & . & . & . \\
	=2 & =2 & =2 & =2 \\
	iv & c & ib & c \\
	. & . & . & . \\
	. & . & V7c & g \\
	. & . & . & . \\
	ib & g & . & . \\
	. & . & . & . \\ \hline
	\end{tabular}
	\caption{Sample evaluation file.}
	\label{table:evaluation_file}
	\end{table}

	\autoref{table:evaluation_file} provides a first glance at a fragment of an evaluation file, extracted from the first two measures of the second movement of the String Quartet Op.20 No.3.

	The first characteristic I want to discuss about this evaluation file, is its normalization of time units.

  \subsection{Normalization of time units}

		\begin{table}[tbp]
		\centering
		\begin{tabular}{|c|c|c|c|c|}
		\hline
		Measure & \multicolumn{2}{c|}{Analysis 1} & \multicolumn{2}{c|}{Analysis 2} \\ \hline
		\multirow{2}{*}{Anacrusis} & . & . & . & . \\ \cline{2-5}
		 & . & . & . & . \\ \hline
		\multirow{3}{*}{First measure} & i & g & . & . \\ \cline{2-5}
		 & iio & a & iv & f \\ \cline{2-5}
		 & ib & g & . & . \\ \hline
		\multirow{3}{*}{Second measure} & iv & c & ib & c \\ \cline{2-5}
		 & . & . & V7c & g \\ \cline{2-5}
		 & ib & g & . & . \\ \hline
		\end{tabular}
		\caption{Original time units}
		\label{table:no_normalization}
		\end{table}

		If the evaluation file displayed before would keep the original time units of the labels, it would look like as shown in \autoref{table:no_normalization}.

		One problem that arises with the way **harm spines are appended to the scores, is that these spines do not store any explicit information about time duration. The labels are properly interpreted in the full score because they are aligned to the note values from the **kern spines, which do have time duration information.

		In order to avoid this and other problems related to duration, during the creation of the evaluation file, the time units of the score get normalized to the shortest note of the score.

	\subsection{Finding the shortest note}
		In order to find the shortest note, the \emph{census} program is being used, which is part of the \emph{Humdrum Toolkit} \cite{humdrum}.

		\begin{table}[tbp]
		\centering
		\begin{tabular}{ll}
		HUMDRUM DATA &  \\
		 &  \\
		Number of data tokens: & 1780 \\
		Number of null tokens: & 515 \\
		Number of multiple-stops: & 1 \\
		Number of data records: & 445 \\
		Number of comments: & 27 \\
		Number of interpretations: & 7 \\
		Number of records: & 479 \\
		 &  \\
		KERN DATA &  \\
		 &  \\
		Number of note-heads: & 785 \\
		Number of notes: & 741 \\
		Longest note: & 2 \\
		Shortest note: & 8 \\
		Highest note: & ddd \\
		Lowest note: & DD \\
		Number of rests: & 125 \\
		Maximum number of voices: & 5 \\
		Number of single barlines: & 88 \\
		Number of double barlines: & 1
		\end{tabular}
		\caption{Output from the census program of the Humdrum Toolkit}
		\label{table:census}
		\end{table}

		\autoref{table:census} shows the output of the \emph{census} program for the humdrum file of the second movement of Op.20 No.3. Among the output of the \emph{KERN DATA} section, the duration of the shortest note of the humdrum score is displayed. This note duration is parsed from the output of the program and used to make a new timebase for the evaluation file.

		As a matter of explanation, the string value that appears as the shortest note is in the frequently used reciprocal notation. In this case, a value of "8" means the note duration is 1/8 of a whole note, or an eight note. Basically, as this example holds a time signature of 3/4 and the shortest note is an eight note, six eight notes are expected for every measure.

		% Please add the following required packages to your document preamble:
		% \usepackage{multirow}
		\begin{table}[tbp]
		\centering
		\begin{tabular}{|c|c|c|c|c|}
		\hline
		Measure & \multicolumn{2}{c|}{Analysis 1} & \multicolumn{2}{c|}{Analysis 2} \\ \hline
		\multirow{6}{*}{Anacrusis} & . & . & . & . \\ \cline{2-5}
		 & . & . & . & . \\ \cline{2-5}
		 & . & . & . & . \\ \cline{2-5}
		 & . & . & . & . \\ \cline{2-5}
		 & . & . & . & . \\ \cline{2-5}
		 & . & . & . & . \\ \hline
		\multirow{6}{*}{First} & i & g & . & . \\ \cline{2-5}
		 & . & . & . & . \\ \cline{2-5}
		 & iio & a & iv & f \\ \cline{2-5}
		 & . & . & . & . \\ \cline{2-5}
		 & ib & g & . & . \\ \cline{2-5}
		 & . & . & . & . \\ \hline
		\multirow{6}{*}{Second} & iv & c & ib & c \\ \cline{2-5}
		 & . & . & . & . \\ \cline{2-5}
		 & . & . & V7c & g \\ \cline{2-5}
		 & . & . & . & . \\ \cline{2-5}
		 & ib & g & . & . \\ \cline{2-5}
		 & . & . & . & . \\ \hline
		\end{tabular}
		\caption{Setting time units normalization to the evaluation file}
		\label{table:normalization}
		\end{table}

		\autoref{table:normalization} shows precisely this, the normalized evaluation file, six entries per measure, each of them corresponding to an eight note, the shortest note of the music score.

\section{Computing a result from an evaluation file}
	Once the contents and characteristics of an evaluation file have been described, now I will describe the process of comparing between both analyses contained in this file to obtain a final result of similarity between the analyses.

	\subsection{Comparing root spines}
		The comparison process could be summarized simply as counting the number of string matches between the second and fourth spines of the evaluation file, with some minor considerations.

		As long as one root appears in one of the spines, it will be considered the "current" root, meaning that any NULL record (a record with a dot in it) is considered to be of the same root as the last one appearing in the spine.

		\begin{table}[tbp]
		\centering
		\begin{tabular}{l|ll|ll|}
		\cline{2-5}
		 & \multicolumn{2}{l|}{How it is written} & \multicolumn{2}{l|}{How it is interpreted} \\ \hline
		\multicolumn{1}{|l|}{Time unit} & \multicolumn{1}{l|}{Root 1} & Root 2 & \multicolumn{1}{l|}{Root 1} & Root 2 \\ \hline
		\multicolumn{1}{|l|}{1} & . & . & . & . \\ \cline{1-1}
		\multicolumn{1}{|l|}{2} & . & . & . & . \\ \cline{1-1}
		\multicolumn{1}{|l|}{3} & . & . & . & . \\ \cline{1-1}
		\multicolumn{1}{|l|}{4} & . & . & . & . \\ \cline{1-1}
		\multicolumn{1}{|l|}{5} & . & . & . & . \\ \cline{1-1}
		\multicolumn{1}{|l|}{6} & . & . & . & . \\ \cline{1-1}
		\multicolumn{1}{|l|}{7} & g & . & g & . \\ \cline{1-1}
		\multicolumn{1}{|l|}{8} & . & . & g & . \\ \cline{1-1}
		\multicolumn{1}{|l|}{9} & a & f & a & f \\ \cline{1-1}
		\multicolumn{1}{|l|}{10} & . & . & a & f \\ \cline{1-1}
		\multicolumn{1}{|l|}{11} & g & . & g & f \\ \cline{1-1}
		\multicolumn{1}{|l|}{12} & . & . & g & f \\ \cline{1-1}
		\multicolumn{1}{|l|}{13} & c & c & c & c \\ \cline{1-1}
		\multicolumn{1}{|l|}{14} & . & . & c & c \\ \cline{1-1}
		\multicolumn{1}{|l|}{15} & . & g & c & g \\ \cline{1-1}
		\multicolumn{1}{|l|}{16} & . & . & c & g \\ \cline{1-1}
		\multicolumn{1}{|l|}{17} & g & . & g & g \\ \cline{1-1}
		\multicolumn{1}{|l|}{18} & . & . & g & g \\ \hline
		\end{tabular}
		\caption{Root spines, the 2nd and 4th spines of an evaluation file}
		\label{table:roots}
		\end{table}

		This becomes clearer from looking at \autoref{table:roots}. The \emph{NULL} records are replaced by the last root that appeared in the spine.

		After this assumption has been explained, the next step is simply comparing the roots from both analyses, row by row.

	\subsection{Harmonic roots}
		\begin{table}[tbp]
		\centering
		\begin{tabular}{|l|l|l|l|}
		\hline
		Time unit & Root 1 & Root 2 & Match \\ \hline
		1 & . & . & Yes \\ \hline
		2 & . & . & Yes \\ \hline
		3 & . & . & Yes \\ \hline
		4 & . & . & Yes \\ \hline
		5 & . & . & Yes \\ \hline
		6 & . & . & Yes \\ \hline
		7 & g & . & No \\ \hline
		8 & g & . & No \\ \hline
		9 & a & f & No \\ \hline
		10 & a & f & No \\ \hline
		11 & g & f & No \\ \hline
		12 & g & f & No \\ \hline
		13 & c & c & Yes \\ \hline
		14 & c & c & Yes \\ \hline
		15 & c & g & No \\ \hline
		16 & c & g & No \\ \hline
		17 & g & g & No \\ \hline
		18 & g & g & No \\ \hline
		\end{tabular}
		\caption{Matching the root spines from an evaluation file}
		\label{table:matches}
		\end{table}

		Little has been said about the **root elements corresponding to the second and fourth spines, and where have they been obtained from.

		These roots are computed by using \emph{harm2kern}, another tool from Humdrum Extras \cite{humextra}. The idea behind using these \emph{root} values in the evaluation, instead of for example, a raw string comparison between the roman numerals, is to resolve the ambiguity that is caused by possible different tonalities in the harmonic analyses.

		Due to the nature of roman numerals, they might be representing something similar in context, but quite different in spelling, these roots pretend to "normalize" a roman numeral label and resolve it into a single letter.

		There are of course drawbacks out of this approach, the most important of them being that if the analysis considers changes in tonality during the music score, these are ignored. Ideally, this changes in tonality should not be ignored, and they comprehend an important part of the analysis, however, proposing an evaluation process that incorporates these changes in tonality is rather difficult, and it is proposed as one of the key aspects to improve in future work.

		Finally, \autoref{table:matches} displays the evaluation summary for the first two measures (and anacrusis) of the example shown at the beginning of the chapter. This example consists of 18 time units, 8 of them matching. However, only 2 of them really match due to their root value, the rest are simply the initial padding of NULL records.

\newpage
