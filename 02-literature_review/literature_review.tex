\chapter{Literature review}

\section{Harmonic analysis in music theory}
  \subsection{Fundamental bass}
  Rameau's theory, the first theory. It was published in 1722, and considers the movement of the bass as the most important harmonic consideration, and from which most harmonic implications can be inferred.
  \subsection{Root succession tables}
  Appears in several textbooks, e.g., Walter Piston's Harmony book. Not really based in scientific data, but somewhat validated by scientific research now.
  \subsection{Figured bass}
  Used mainly for performance. The least common target output for automatic approaches.
  \subsection{Functional harmony}
  Based in Hugo Riemann's theory, it is the focus of this work. It contains more information about tonal context than fundamental bass or root succession theories.
\section{Harmonic analysis in computing}
  Automatic harmonic analysis can be classified in different ways. Most researchers agree that the pioneer work in this problem was the approach from Winograd in 1968. Since then, researchers have developed this idea in very different ways. I will discuss first the classical music approaches versus the Jazz harmonic analysis approaches. Then I will discuss how the problem has also been addressed in very different ways, and with different purposes in mind. Finally, I will review some of the musicological aspects of the string quartets, Haydn as a composer of classical musical forms, and finally I will discuss the specific string quartet that was manually annotated for this work, its musicological context and so on.
  \subsection{Classical vs. Jazz}
    \subsubsection{Differences}
    It could be said that the main difference between analyzing classical music or jazz are musical considerations. The harmonic common practices and dictionary of chords are different. However, hierarchically, both types of music are tonal.
    \subsubsection{Similarities}
    There are systems that claim to be agnostic of the music style, these models are usually based on pitch-class sets. Other common ground between classical music and jazz lies in the approaches based in grammars.
  \subsection{Type of model}
  Some other way to classify approaches is by the underlying technology used for the analysis
    \subsubsection{Rule-based}
    It is one of the first approaches
    \subsubsection{Probabilistic}
    Most of the rule-based approaches evolved towards probabilistic models. One important example is the Temperley algorithm that is used for this work, which was eventually developed as a Bayesian model in the second version of the Melisma Music Analyzer. However, this has never made out all the way to functional harmonic analysis, that is why I am remaining with the first approach.
    \subsubsection{Grammar-based}
    The first approach, from Winograd is a grammar based approach. Recently, these approaches have been used to find hierarchical relationships between chords.
  \subsection{Target output}
  Probably the most "musical" way of classifying the harmonic analysis, is by the target output the model is expected to produce.
    \subsubsection{Figured bass}
    Barthelemy is the only one.
    \subsubsection{Functional harmonic analysis}
    Temperley-Sleator-Sapp, Raphael, Illescas
    \subsubsection{Tonal hierarchies}
    Rohrmeier, Bas de Haas
\section{String quartets}
  \subsection{Musical form in harmony}
\section{Joseph Haydn}
  \subsection{Who was}
  \subsection{What music does he represent}
  \subsection{His string quartets}
    \subsubsection{Op.20, Sun quartets}

\newpage
