\normallinespacing

\chapter{Literature review}

\section{Music theory}
  \subsection{Fundamental bass}
  Rameau\'s theory, the first theory
  \subsection{Root succession tables}
  Appears in several textbooks, e.g., Walter Piston's Harmony book. Not really based in scientific data, but somewhat validated by scientific research now.
  \subsection{Figured bass}
  Used mainly for performance
  \subsection{Functional harmony}
  Based in Hugo Riemann\'s theory, it is the focus of this work. It contains more information about tonal context than fundamental bass or root succession theories.

\section{Models}
  \subsection{Classical vs. Jazz}
    \subsubsection{Differences}
    \subsubsection{Similarities}
  \subsection{Type of model}
    \subsubsection{Rule-based}
    \subsubsection{Probabilistic}
    \subsubsection{Grammar-based}
  \subsection{Target output}
    \subsubsection{Figured bass}
    Barthelemy is the only one.
    \subsubsection{Functional harmonic analysis}
    Temperley-Sleator-Sapp, Raphael, Illescas
    \subsubsection{Tonal hierarchies}
    Rohrmeier, Bas de Haas
\section{String quartets}
  \subsection{Musical form in harmony}
\section{Joseph Haydn}
  \subsection{Who was}
  \subsection{What music does he represent}
  \subsection{His string quartets}
  \subsection{Op.20, Sun quartets}

\newpage
