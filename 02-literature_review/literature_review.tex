\chapter{Literature review}
\section{Harmonic analysis in music theory}
Harmonic analysis could be seen from different perspectives, the first one I would like to address is how it was conceived and modeled by music theorists. Starting with the french composer Jean-Philippe Rameau (1683-1764), until the theory from the german theorist Hugo Riemann (1849-1919) in the late nineteenth century.
  \subsection{Fundamental bass}
  The original idea by Rameau was based on the movement of the bass note, the so called \emph{Fundamental Bass}, which constructed rules and principles of how the movement of a bass note determined also the movement of the harmony. Rameau's theory limited the dictionary of chords to triads and some special cases of seventh chords, and considered chord inversions for the first time, from which the core concept of harmonic root arises and suggests that the bass note could be serving a harmonic root located in the superior voices. This theory is pioneer in separating the melodic movement and coincidence in time from counterpoint \cite{beach1974origins}, and looking at the music from a vertical perspective, a vision that spread in the following periods of western art music.
  \subsection{Root succession tables}
  These set of theories, denomined by Tymoczko as \emph{scale-degree} theories \cite{tymoczko2001root}, assign a number to each of the degrees in a scale, which represents its triad, and then trying to infer the most common succession of that particular scale degree to another. This theories have been used commonly in Harmony text books, and in principle, the scale-degree transitions have not been obtained scientifically. However, due to the probabilistic nature of these theories, there have been recent efforts in validating their statements and accuracy using computational resources, such as first-order Markov models.
  \subsection{Functional harmony}
  In 1893, Hugo Riemann presented his \emph{Vereinfachte Harmonielehre}, which placed together ideas and theories from himself and earlier theorists, and gave birth to what is called \emph{Functional Harmony}. The most notable, and allegedly controversial, characterization that comes from the functions theory, is the idea of categorization of chords. Functional harmony considers that chords belong to one of three tonal functions:
  \begin{itemize}
    \item Tonic
    \item Subdominant
    \item Dominant
  These functions contain all individual chords, but are essentially represented by the primary triads: I, IV and V.
  (More things to be said about functional harmony...)
  Due to this categorization of chords, functional harmony contains more information about tonal contexts and semantics, and therefore, as an analysis output becomes more interesting than fundamental bass or root succession theories. Hence, the decision of targeting a functional analysis outcome from a computational mean.
\section{Harmonic analysis in computing}
  Automatic harmonic analysis can be classified in different ways. Most researchers agree that the pioneer work in this problem was the approach from Winograd in 1968. Since then, researchers have developed this idea in very different ways. I will discuss first the classical music approaches versus the Jazz harmonic analysis approaches. Then I will discuss how the problem has also been addressed in very different ways, and with different purposes in mind. Finally, I will review some of the musicological aspects of the string quartets, Haydn as a composer of classical musical forms, and finally I will discuss the specific string quartet that was manually annotated for this work, its musicological context and so on.
  \subsection{Classical vs. Jazz}
    \subsubsection{Differences}
    It could be said that the main difference between analyzing classical music or jazz are musical considerations. The harmonic common practices and dictionary of chords are different. However, hierarchically, both types of music are tonal.
    \subsubsection{Similarities}
    There are systems that claim to be agnostic of the music style, these models are usually based on pitch-class sets. Other common ground between classical music and jazz lies in the approaches based in grammars.

    %Scholz 2005
    This working is very oriented to Jazz music, however, the authors claim it was meant to analyze any music whose harmony is functional. The model is done following and extending the ideas from pachet2000computer to recognize modal borrowing and secondary dominants. It was implemented in Java, however, no url is provided to the implementation of the approach.
    %scholz 2005
  \subsection{Type of model}
  Some other way to classify approaches is by the underlying technology used for the analysis
    \subsubsection{Rule-based}
    It is one of the first approaches
    \subsubsection{Probabilistic}

    % Rohrmeier 2008
    This approach also uses MIDI files as input, getting rid of pitch-spelling information, it also seems to be very simplistic as it only extends Melisma in providing major or minor modes for the root analysis. Experiments are done over different kinds of music, which is outside the constraints of classical music that we have set.
    % Rohrmeier 2008

    Most of the rule-based approaches evolved towards probabilistic models. One important example is the Temperley algorithm that is used for this work, which was eventually developed as a Bayesian model in the second version of the Melisma Music Analyzer. However, this has never made out all the way to functional harmonic analysis, that is why I am remaining with the first approach.
    \subsubsection{Grammar-based}
    The first approach, from Winograd is a grammar based approach. Recently, these approaches have been used to find hierarchical relationships between chords.
  \subsection{Target output}
  Probably the most "musical" way of classifying the harmonic analysis, is by the target output the model is expected to produce.
    \subsubsection{Figured bass}
    Barthelemy is the only one.
    \subsubsection{Functional harmonic analysis}
    Temperley-Sleator-Sapp, Raphael, Illescas

    % Raphael
    This model is one of the few pure-functional harmonic analysis approaches, oriented towards the analysis of common-practice music. The idea is simple and the arbitrary assumptions simplify the parameters of the model. Some constant that remains as in previous models is the fact that it gets rid of all the pitch-spelling information and replaced for solely pitch information. This model, unlike Temperley, does not try to reconstruct the pitch spelling information back by any algorithmic means. I believe this information could be added, and in the words of raphael2003harmonic, it is an obvious extension to the model.
    % / Raphael

    \subsubsection{Tonal hierarchies}
    Rohrmeier, Bas de Haas
\section{String quartets}
  \subsection{Musical form in harmony}
\section{Joseph Haydn}
  \subsection{Who was}
  \subsection{What music does he represent}
  \subsection{His string quartets}
    \subsubsection{Op.20, Sun quartets}

\newpage
