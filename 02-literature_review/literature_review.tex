\chapter{Literature review}
\section{Harmonic analysis in music theory}
Harmonic analysis could be seen from different perspectives, the first one I would like to address is how it was conceived and modeled by music theorists. Starting with the french composer Jean-Philippe Rameau (1683-1764), until the theory from the german theorist Hugo Riemann (1849-1919) in the late nineteenth century.
  \subsection{Fundamental bass}
  The original idea by Rameau was based on the movement of the bass note, the so called \emph{Fundamental Bass}, which constructed rules and principles of how the movement of a bass note determined also the movement of the harmony. Rameau's theory limited the dictionary of chords to triads and some special cases of seventh chords, and considered chord inversions for the first time, from which the core concept of harmonic root arises and suggests that the bass note could be serving a harmonic root located in the superior voices. This theory is pioneer in separating the melodic movement and coincidence in time from counterpoint \cite{beach1974origins}, and looking at the music from a vertical perspective, a vision that spread in the following periods of western art music.
  \subsection{Root succession tables}
  These set of theories, denomined by Tymoczko as \emph{scale-degree} theories \cite{tymoczko2001root}, assign a number to each of the degrees in a scale, which represents its triad, and then trying to infer the most common succession of that particular scale degree to another. This theories have been used commonly in Harmony text books, and in principle, the scale-degree transitions have not been obtained scientifically. However, due to the probabilistic nature of these theories, there have been recent efforts in validating their statements and accuracy using computational resources, such as first-order Markov models.
  \subsection{Functional harmony}
  In 1893, Hugo Riemann presented his \emph{Vereinfachte Harmonielehre}, which placed together ideas and theories from himself and earlier theorists, and gave birth to what is called \emph{Functional Harmony}. The most notable, and allegedly controversial, characterization that comes from the functions theory, is the idea of categorization of chords. Functional harmony considers that chords belong to one of three tonal functions:
  \begin{itemize}
    \item Tonic
    \item Subdominant
    \item Dominant
  \end{itemize}
  These functions contain all individual chords, but are essentially represented by the primary triads: I, IV and V.
  Due to this categorization of chords, functional harmony contains more information about tonal contexts and semantics, and therefore, as an analysis output becomes more interesting than fundamental bass or root succession theories.
\section{Harmonic analysis in computing}
  Automatic harmonic analysis can be classified in different ways. Most researchers agree that the pioneer work in this problem was the approach from Winograd in 1968 \cite{winograd1968linguistics}. Since then, researchers have developed this idea in very different ways. I will discuss first the classical music approaches versus the jazz harmonic analysis approaches. Then I will discuss some of the methods and models used for harmonic analysis problems, which may have quite different purposes in mind. Finally, I will review some of the musicological aspects of the string quartets, Haydn as a composer of classical musical forms, and finally I will discuss the specific string quartet that was manually annotated for this work, its musicological context and so on.
  \subsection{Target musical style}
  Tonal music could comprehend a wide range of music as early baroque's Claudio Monteverdi to 20th century jazz fusion's Allan Holdwsworth. In that sense, an automatic harmonic analysis approach should be able to reach and point out interesting things about all these kinds of music. In practice, however, it is difficult that one approach could serve all sorts of tonal music and at the same time deal with the details, common practices and corner cases from these different music styles and periods. Researchers have systematically split their efforts to target specific kinds of music, and even sometimes, specific composers.
  Among the approaches that cover different kinds of music, there is the General Chord Type (GCT) representation from Emilios Cambouropoulos \cite{cambouropoulos2014idiom}, which is intended to be used in the universal classification of chord verticals according to the idiom that is trying to model. This model is not restricted to tonal music, having the possibility of working in atonal music as well, however, as I stated before, such a flexible model does not deal with the complex and specific scenarios of tonal music, in this case, simply because it is not designed to do it.
  The natural separation that has emerged among researches is to either tackle harmonic analysis targetting classical music or jazz music. As the interest of my work is classical music, I will only briefly describe some of the pillar works in the context of jazz harmonic analysis.
    \subsubsection{Jazz harmonic analysis}
    After Winograd, probably one of the pillar works in harmonic analysis for jazz is that of John Wade Ulrich \cite{ulrich1977analysis}. This work developed a functional analysis, identifying the function of each chord in a musical piece. The input for this model was a sequence of chords and it incorporated the detection of keys. Among some of the restrictions it has is that it does not work for music in a minor tonality, and constraints chord changes to occur once every beat.
    Following to the model from Ulrich, probably one of the next pillars in harmonic analysis oriented to jazz music and a direct successor to the baseline of Ulrich is the model from Francois Pachet \cite{pachet2000computer} who presents a hierarchical model to analyze jazz harmonic progressions, similarly taking as input a sequence of chords and deriving a hierarchical description of modulations. This model does not consider voice-leading and lands in the more "hierarchical" type of analysis, which give structure to a set of chord labels. It was later extended by Ricardo Scholz \cite{scholz2005automating} who attempted to incorporate modal borrowings and secondary dominants.
    From these research, we can infer an important fact, most of the jazz oriented approaches lie within the context of \emph{hierarchical} approaches to functional harmony, and oftenly trivialize the input to chord labels to focus on the functional hierarchies themselves. One of the most important problems in classical music, for example, is precisely in how to deal with all these melodic streams before they can be considered chord labels. This happens maybe because the harmonic movement in Jazz is much more aggressive and departs from a tonal center with ease, making the effort of tracking fast-moving tonal centers an attractive problem, and melodic, polyphonical implications of the music less useful, whereas in classical music it is usual to well-establish one tonal center before moving to the next, at least for the classical period, but extracting the implied tonal function within the different melodic streams is not a trivial task to solve. These characteristics of the music styles probably point out and justify the motivations for separating the models according to the music that is being analyzed.
    As the target music of this work belongs to the early Classical period in western art music, I will be dealing with the approaches concerning this kind of music, with special attention to those approaches that attempt to solve the problem entirely, without assuming an input of chord labels, but a full musical score.
  \subsection{Harmonic analysis chronology}
  Once established that this work is dealing with music from the Classical period and therefore its interest is in the approaches that attempt to provide an automatic harmonic analysis of such subset of tonal music, I will start to describe some of the remarkable efforts done, putting special emphasis to their contributions, inputs, outputs and flaws.
    \subsubsection{The pioneer work by Terry Winograd}
    As it was stated before, researchers in the field mostly mention the approach from Terry Winograd, in 1968, as the pioneer work in the field. This work is not only important because it is the first and pioneering work in computing a functional harmonic analysis, but also because it linked the computational techniques used in natural language processing to music. The model from Winograd was evaluated over music from Johann Sebastian Bach, and pretends to output a functional harmonic analysis of such pieces of music. To provide an output, it requires a preliminary hand-made conversion of the original score and turn it into a sequence of four-part perfect chords. This allowed him to process a score using his implementation in the LISP programming language, but it also means that during this pre-processing stage the non-harmonic tones are eliminated before solving the problem. In his 1997 harmonic analysis algorithm \cite{temperley1997algorithm}, David Temperley provided insight about the flaws of Winograd's model, among them he mentions the issues concercing melodic seggregation and arpeggiations.
    \subsubsection{Expert system from Maxwell}
    A direct successor of Winograd's approach, the model from John Maxwell, which was part of his PhD dissertation in 1984, and successively published in 1992 \cite{maxwell1992expert}, is probably the best example of the rule-based approaches towards harmonic analysis. Same as Winograd, Maxwell's target was to output a functional harmonic analysis from a music score. The model from Maxwell has fifty five rules that pretend to reduce the vertical sonorities into a chord sequence, and then deciding for key changes. Some of the rules are intuitive and basic, e.g., \emph{"Perfect and imperfect consonant intervals constitute a consonant interval. Every other is a dissonant interval."}, while others appear cryptic and difficult to understand, e.g., \emph{"If the goal chord falls on a strong beat and it is a major triad or major-minor seventh, and the root movement from the pre-cadence is an ascending or descending perfect fifth or major second or a descending minor second, and when the root motion is a descending fifth, the pre-cadence is not a potential dominant, and when the root motion is an ascending fifth the pre-cadence is triadic, then the pseudo-cadence is a half cadence, and its strength increases by 10."}.
    The later rule also reveals a problem that was pointed out by future researchers, the use of arbitrary, fixed values while determining the strength of a cadence. Even so, the results of Maxwell's approach get really close to the outcome expected by a music theorist analyzing a music score and determining its functional harmony. This work was tested over three different movements of Johann Sebastian Bach's Six French Suites: The sarabande from Suite No.1, the menuet from Suite No.2 and the gavotte from Suite No.5. The pieces selected comprehend different problems and levels of complexity to be addressed: Four-part harmony with several non-harmonic tones, 2-voice continuous contrapuntual movement, and a varying contrapuntual texture, respectively. David Temperley put the limitations of Winograd and Maxwell's approaches pretty close together, summarizing them in the following:
		\begin{itemize}
			\item Sequences of notes that are not displayed simultaneously (vertically), as arpeggiations of chords.
			\item Missing pitches in the spelling of a full chord, which can be deduced from the context.
			\item Ornamental notes. Maxwell proposes specific rules to deal with these notes, but according to Temperley, neither Maxwell’s or Winograd’s are good enough to correctly detect ornamental notes.
    \end{itemize}

    % Rohrmeier 2008
    This approach also uses MIDI files as input, getting rid of pitch-spelling information, it also seems to be very simplistic as it only extends Melisma in providing major or minor modes for the root analysis. Experiments are done over different kinds of music, which is outside the constraints of classical music that we have set.
    % Rohrmeier 2008

    Most of the rule-based approaches evolved towards probabilistic models. One important example is the Temperley algorithm that is used for this work, which was eventually developed as a Bayesian model in the second version of the Melisma Music Analyzer. However, this has never made out all the way to functional harmonic analysis, that is why I am remaining with the first approach.
    \subsubsection{Grammar-based}
    The first approach, from Winograd is a grammar based approach. Recently, these approaches have been used to find hierarchical relationships between chords.
  \subsection{Target output}
  Probably the most "musical" way of classifying the harmonic analysis, is by the target output the model is expected to produce.
    \subsubsection{Figured bass}
    Barthelemy is the only one.
    \subsubsection{Functional harmonic analysis}
    Winograd, History aside, the input format for the program is not convenient, and as presented by future researchers, the algorithm has problems with certain cases, such as those involving arpeggiations. It cannot be denied that it this work marked a checkpoint and reference for all future work in harmonic analysis.

    Temperley-Sleator-Sapp, Raphael, Illescas

    % Raphael
    This model is one of the few pure-functional harmonic analysis approaches, oriented towards the analysis of common-practice music. The idea is simple and the arbitrary assumptions simplify the parameters of the model. Some constant that remains as in previous models is the fact that it gets rid of all the pitch-spelling information and replaced for solely pitch information. This model, unlike Temperley, does not try to reconstruct the pitch spelling information back by any algorithmic means. I believe this information could be added, and in the words of raphael2003harmonic, it is an obvious extension to the model.
    % / Raphael

    \subsubsection{Tonal hierarchies}
    Rohrmeier, Bas de Haas
\section{String quartets}
  \subsection{Musical form in harmony}
\section{Joseph Haydn}
  \subsection{Who was}
  \subsection{What music does he represent}
  \subsection{His string quartets}
    \subsubsection{Op.20, Sun quartets}

\newpage
